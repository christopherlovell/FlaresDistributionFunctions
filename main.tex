%%%%%%%%%%%%%%%%%%%%%%%%%%%%%%%%%%%%%%%%%%%%%%%%%%
% Basic setup. Most papers should leave these options alone.
\documentclass[fleqn,usenatbib]{mnras}

% MNRAS is set in Times font. If you don't have this installed (most LaTeX
% installations will be fine) or prefer the old Computer Modern fonts, comment
% out the following line
% \usepackage{newtxtext,newtxmath}
% Depending on your LaTeX fonts installation, you might get better results with one of these:
%\usepackage{mathptmx}

%\usepackage{txfonts}

% Use vector fonts, so it zooms properly in on-screen viewing software
% Don't change these lines unless you know what you are doing
\usepackage[T1]{fontenc}
\usepackage{ae,aecompl}
\usepackage{ulem}

%%%%% AUTHORS - PLACE YOUR OWN PACKAGES HERE %%%%%

% Only include extra packages if you really need them. Common packages are:
\usepackage{graphicx}	% Including figure files
\usepackage{amsmath}	% Advanced maths commands
\usepackage{amssymb}	% Extra maths symbols

%%%%%%%%%%%%%%%%%%%%%%%%%%%%%%%%%%%%%%%%%%%%%%%%%%

%%%%% AUTHORS - PLACE YOUR OWN COMMANDS HERE %%%%%

% Please keep new commands to a minimum, and use \newcommand not \def to avoid
% overwriting existing commands. Example:
%\newcommand{\pcm}{\,cm$^{-2}$}	% per cm-squared

\newcommand{\aswin}[1]{{\color{green}#1}}
\newcommand{\peter}[1]{{\color{magenta}#1}}
\newcommand{\chris}[1]{{\color{cyan}#1}}
\newcommand{\todo}[1]{{\color{green}***#1***}}
\newcommand{\changed}[2]{{\color{red}\sout{#1}}{\color{blue}#2}}

% Abbreviated referencing macros
\newcommand{\app}[1]{Appendix~\ref{sec:#1}}
\newcommand{\eq}[1]{Equation~\ref{eq:#1}}
\newcommand{\fig}[1]{Figure~\ref{fig:#1}}
\renewcommand{\sec}[1]{Section~\ref{sec:#1}}
\newcommand{\tab}[1]{Table~\ref{tab:#1}}
\newcommand{\App}[1]{Appendix~\ref{sec:#1}}
\newcommand{\Eq}[1]{Equation~\ref{eq:#1}}
\newcommand{\Fig}[1]{Figure~\ref{fig:#1}}
\newcommand{\Sec}[1]{Section~\ref{sec:#1}}
\newcommand{\Tab}[1]{Table~\ref{tab:#1}}
% Names of things
\newcommand{\bluetides}{\mbox{\sc{BlueTides}}}
\newcommand{\ceagle}{\mbox{\sc{C-Eagle}}}
\newcommand{\eagle}{\mbox{\sc{Eagle}}}
\newcommand{\lgals}{\mbox{\sc{L-Galaxies}}}
\newcommand{\euclid}{\mbox{\it Euclid}}
\newcommand{\flares}{\mbox{\sc Flares}}
\newcommand{\flare}{\mbox{\sc Flare}}
\newcommand{\wfirst}{\mbox{\it WFIRST}}
\newcommand{\jwst}{\mbox{\it JWST}}
% Other abbreviations
%\newcommand{\SFR}{\mathrm{SFR}}
\newcommand{\SFR}{\psi}
\newcommand{\zoom}{\mbox{zoom}}
%\newcommand{\zoom}{\mbox{'zoom'}}
% Units
\newcommand{\Lsun}{\mbox{L$_\odot$}}
\newcommand{\Msun}{\mbox{M$_\odot$}}
\newcommand{\cMpch}{h$^{-1}$ cMpc}

%%%%%%%%%%%%%%%%%%%%%%%%%%%%%%%%%%%%%%%%%%%%%%%%%%

%%%%%%%%%%%%%%%%%%% TITLE PAGE %%%%%%%%%%%%%%%%%%%

% Title of the paper, and the short title which is used in the headers.
% Keep the title short and informative.
\title[FLARES I]{First Light And Reionisation Epoch Simulations \\(FLARES) I: Environmental Dependence of High-Redshift Galaxy Evolution}

% The list of authors, and the short list which is used in the headers.
% If you need two or more lines of authors, add an extra line using \newauthor
\author[C. C. Lovell et al.]{Christopher C. Lovell,$^{1,2}$\thanks{E-mail: c.lovell@herts.ac.uk (CCL)}
Aswin P. Vijayan,$^{2}$
Peter A. Thomas,$^{2}$
\newauthor
Stephen M. Wilkins,$^{2}$
David J. Barnes,$^{3}$
Dimitrios Irodotou,$^{2}$
Will Roper$^{2}$
\\
% List of institutions
$^{1}$Centre for Astrophysics Research, School of Physics, Astronomy \& Mathematics, \\University of Hertfordshire, Hatfield AL10 9AB, UK\\
$^{2}$Astronomy Centre, University of Sussex, Falmer, Brighton BN1 9QH, UK\\
$^{3}$Department of Physics, Kavli Institute for Astrophysics and Space Research, Massachusetts Institute of Technology,\\Cambridge, MA 02139, USA
}

% These dates will be filled out by the publisher
\date{Accepted XXX. Received YYY; in original form ZZZ}

% Enter the current year, for the copyright statements etc.
\pubyear{2020}

% Don't change these lines
\begin{document}
\label{firstpage}
\pagerange{\pageref{firstpage}--\pageref{lastpage}}
\maketitle

% Abstract of the paper
\begin{abstract}
We introduce the First Light and Reionisation Epoch Simulations (\flares), a suite of \zoom\ simulations using the \textsc{Eagle} model.
We re-simulate a range of overdensities during the Epoch of Reionisation (EoR) in order to build composite distribution functions, as well as explore the environmental dependence of galaxy formation and evolution during this critical period of galaxy assembly.
The regions are selected from a large $(3.2 \;\mathrm{cGpc})^{3}$ parent volume, based on their overdensity within a sphere of radius $14$ \cMpch.
We then re-simulate with full hydrodynamics, and employ a novel weighting scheme that allows the construction of composite distribution functions that are representative of the full parent volume.
This significantly extends the dynamic range compared to smaller volume periodic simulations.
We present an analysis of the galaxy stellar mass function, the star formation rate distribution function and the star forming sequence predicted by \flares, and compare to a number of observational and model constraints.
We also analyse the environmental dependence over an unprecedented range of overdensity.
This increased dynamic range will allow us to make predictions for a number of large area surveys that will probe the EoR in coming years, such as \wfirst\ and \euclid.
\end{abstract}

% Select between one and six entries from the list of approved keywords.
% Don't make up new ones  .
\begin{keywords}
keyword1 -- keyword2 -- keyword3
\end{keywords}

%%%%%%%%%%%%%%%%%%%%%%%%%%%%%%%%%%%%%%%%%%%%%%%%%%

%%%%%%%%%%%%%%%%% BODY OF PAPER %%%%%%%%%%%%%%%%%%

\input intro
\input methods
\input results
%\input disc
\input conc

%%%%%%%%%%%%%%% ACKNOWLEDGEMENTS %%%%%%%%%%%%%%%%%%
\section*{Acknowledgements}
We wish to thank David Barnes, Scott Kay and Adrian Jenkins for their invaluable help getting up and running with the \eagle\ resimulation code.
Thanks also to Rob Crain for helpful discussions.

This work used the DiRAC@Durham facility managed by the Institute for Computational Cosmology on behalf of the STFC DiRAC HPC Facility (www.dirac.ac.uk).
The equipment was funded by BEIS capital funding via STFC capital grants ST/K00042X/1, ST/P002293/1, ST/R002371/1 and ST/S002502/1, Durham University and STFC operations grant ST/R000832/1.
DiRAC is part of the National e-Infrastructure.
Much of the data analysis was undertaken on the {\sc Apollo} cluster at the University of Sussex.

PAT %(ORCID 0000-0001-6888-6483)
acknowledges support from the Science and Technology Facilities Council (grant
number ST/P000525/1).
Lovell acknowledges support from the Royal Society under grant RGF/EA/181016.

%%%%%%%%%%%%%%%%%%%%%%%%%%%%%%%%%%%%%%%%%%%%%%%%%%
%%%%%%%%%%%%%%%%%%%% REFERENCES %%%%%%%%%%%%%%%%%%

\bibliographystyle{mnras}
\bibliography{resims}

%%%%%%%%%%%%%%%%%%%%%%%%%%%%%%%%%%%%%%%%%%%%%%%%%%
%%%%%%%%%%%%%%%%% APPENDICES %%%%%%%%%%%%%%%%%%%%%

\appendix

\input appendix

% \section{Some extra material}

% If you want to present additional material which would interrupt the flow of the main paper,
% it can be placed in an Appendix which appears after the list of references.

%%%%%%%%%%%%%%%%%%%%%%%%%%%%%%%%%%%%%%%%%%%%%%%%%%


% Don't change these lines
\bsp	% typesetting comment
\label{lastpage}
\end{document}

% End of mnras_template.tex
