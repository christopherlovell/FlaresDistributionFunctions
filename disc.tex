\section{Discussion}
\label{sec:conc}


\peter{This section is unnecessary.}
\chris{either we fold some of this in to each section, or I think we should have some discussion here. Even if it is speculative, it gives some tentative interpretation.}

\todo{discuss use of different SFR tracers}
\todo{discuss how re-simulation technique can be improved by adding more regions}

\subsection{Shape of the GSMF}

Galaxies central black holes grow in tandem with their stellar mass, until they are sufficiently massive that their energetic feedback can suppress star formation.
The galaxy then experiences lower star formation, remaining at a similar mass for an extended period.
This galactic `pile-up' explains the increased normalisation at the characteristic mass.

The evolution of the characteristic mass with redshift can be explained by the AGN feedback model in \eagle.
\peter{How?  Why would it not lead to a fixed $M^*$?}

\chris{Could plot stellar mass - black hole mass relation, to show this clearer.}
\peter{Actually, not at all clear to me that this has anything at all to do with black hole feedback.  I would be tempted to delete all the above discussion relating to black holes/AGN as it is just speculation.}

\subsection{The shape of the SFRF}
\chris{some thoughts on this: do we think the exponential turn off of the SFRF is also due to AGN feedback? If so, why doesn't the characteristic SFR evolve with redshift - the normalisation of the main sequence evolves pretty significantly over this same redshift range. If not, perhaps we attribute it to stellar feedback rapidly curtailing SFR in very active systems, i.e. there is a very short duty cycle of intense, > 100 Msol / yr star formation.}
\peter{Actually, this is much more as I would have expected -- that the turnoff mass is independent of redshift.  SO that might be interesting if so.  That would suggest that the turn-off mass for the GSMF is affected by growth due to mergers.
It could get very messy if it is more like what Chris says though.}

