\section{Conclusions}
We have presented the first results from the \flares\ simulations, re-simulations with full hydrodynamics of a range of overdensities during the Epoch of Reionisation (EoR, $z \geqslant 5$) using the \eagle\ \citep{schaye_eagle_2015} physics.
We described our novel weighting procedure that allows the construction of composite distribution functions that mimic extremely large periodic volumes, significantly extending the dynamic range without incurring prohibitively large computational expense.
To demonstrate we presented results for the galaxy stellar mass function (GSMF), the star formation rate distribution function (SFRF) and the star-forming sequence (SFS: SFR versus $M_*$).
Our findings are as follows:

\begin{itemize}
	\item The \flares\ GSMF exhibits a clear double-Schechter shape up to $z = 10$.
				Fits assuming this form show an increasing normalisation, shallower low-mass slope and higher characteristic turnover mass with decreasing redshift.
				The GSMF is in good agreement with observational constraints at all redshifts up to $z = 8$, at which point there is some tension at the knee of the distribution.
    		The normalisation, and to a lesser extent the shape, of the GSMF shows a strong environmental dependence (i.e.~bias).
  \item The SFRF also exhibits a clear double-Schechter shape in the high-SFR regime.
				As for the GSMF, the normalisation increases and the low-mass slope decreases with decreasing redshift; however the characteristic turnover mass varies only weakly with redshfit.
  			There is a mild tension with observational results, which tend to more closely resemble power law-like distributions.
				The SFRF shape and normalisation shows a similar environmental dependence to the GSMF.
  \item The SFS shows no obvious dependence on environment.
				The low-mass slope is relatively invariant with redshift, whereas the high mass slope decreases with decresing redshift.
				The characteristic turnover mass increases slowly with decreasing redshift, and the normalisation decreases by about a factor of 3 between redshifts 10 and 5.
				There is reasonably good agreement with observational constraints at $z = 5-6$.
\end{itemize}

Upcoming space based observatories, such as JWST, \euclid\ and \wfirst\ will provide further probes of the GSMF and SFRF up to $z = 10$.
The large volumes probed by \euclid\ and \wfirst\ in particular will provide stronger constraints on those extreme galaxies that populate the high-mass\,/\,high-SFR tails of each distribution.
Our weighting scheme provides a means of testing the latest, high resolution hydrodynamic simulations against such constraints.  We will also be able to test the impact of cosmic variance on these large surveys.
